\documentclass[a4paper]{norm-resume}

\usepackage{array}
\usepackage{tabularx}
\usepackage{setspace}

\def\changemargin#1#2{\list{}{\rightmargin#2\leftmargin#1}\item[]}
\let\endchangemargin=\endlist 

\pagenumbering{gobble}

\begin{document}

%----------------------------------------------------------------------------------------
%	TITLE SECTION
%----------------------------------------------------------------------------------------

\lastupdated % Print the Last Updated text at the top right

\namesection{Nishant}{Rai}{}{Email: nishantr@iitk.ac.in}

\vspace{2mm}

% Education section : With current and previous schooling

\section{Education \hrulefill}
		
\begin{tabularx}{\textwidth}{c l l r}	
	{April 2017 \hspace{10mm}} & B.Tech \emphasize{(Computer Science And Engineering)} \hspace{22mm} & \emphasize{IIT Kanpur \hspace{10mm}} & {\textbf{{9.9/10.0}}}\\		
	{April 2013 \hspace{10mm}} & Class XII \emphasize{(Central Board for Senior Education)} & \emphasize{K.V. Delhi} & {\textbf{{96.20 \%}}}\\
	{April 2011 \hspace{10mm}} & Class X \emphasize{(Central Board for Senior Education)} & \emphasize{K.V. Shillong} & {\textbf{{10.0/10.0}}}\\	
\end{tabularx}

\vspace{2mm}	% Some space before next section

% Academic Acheivements

\section{Academic Achievements \hrulefill}

\vspace{3mm} %Some space
		
\begin{tightitemize}
	\item Received \textbf{MITACS Globalink Award} for a Summer Research Internship at \textbf{U.B.C.} for the year \textbf{2016}.
	\item Received \textbf{Charpak Research Scholarship} for the year \textbf{2015}.
	\item Received the \textbf{Academic Excellence Award} for exceptional academic performance in \textbf{13-14} and \textbf{14-15} academic session.
	\item Secured AIR - \textbf{257} in {\textbf{JEE (Advanced) 2013}} and AIR - {\textbf{79}} in \textbf{JEE (Mains) 2013}.
	\item One amongst the \textbf{6 INMO Awardees} selected for \textbf{IMO training camp}, after clearing \textbf{INMO '13}.
	\item Awarded Gold medal for being selected for \textbf{IPhO training camp}, after clearing \textbf{INPhO '13}.
	\item Awarded \textbf{Ashray Hasta Award and Scholarship, 2013} for exceptional performance in \textbf{AISSCE, 2013}.
	\item Selected for the prestigious \textbf{KVPY} scholarship, in stream SX.
	\item Secured \textbf{Rank 1} in \textbf{Regional Mathematics Olympiad '12} (Delhi Region).	
\end{tightitemize}
	
\vspace{1mm}	% Some space before next section

% Programming Acheivements

\section{Achievements in Programming \hrulefill}

\vspace{3mm} %Some space

\begin{tightitemize}
	\item Secured \textbf{1\textsuperscript{st}} place in \textbf{Microsoft CodeHunt} amongst individuals from over the country. Selected for \textbf{finals} in \textbf{China}.
	\item Secured \textbf{8\textsuperscript{th}} place in \textbf{ZS Associates Data Science Challenge} amongst \textbf{300+} students from over the country.
	\item Secured \textbf{13\textsuperscript{th}} place in \textbf{IOPC} (International Online Programming Contest) amongst \textbf{900+} teams from over the world held during \textbf{Techkriti '15}.
	\item Secured \textbf{1\textsuperscript{st}} place in the event \textbf{Chaos} (Esoteric Programming Contest) held during \textbf{Techkriti '15}
	\item Secured \textbf{17\textsuperscript{th}} place in in \textbf{Morgan Stanley Codeathon 2014} amongst \textbf{1000+} individuals from over the country.
	\item Secured \textbf{26\textsuperscript{th}} place in \textbf{ACM ICPC Onsite Contest} 2014 amongst \textbf{250+} teams from over the country.
	\item Secured \textbf{1\textsuperscript{st}} place in the \textbf{Web-dev} event during \textbf{Takneek '14} (Inter-Hostel Technical Competition).
	\item Secured \textbf{1\textsuperscript{st}} place in the programming event \textbf{Blackbox} during \textbf{Takneek '13} (Inter-Hostel Technical Competition).
\end{tightitemize}

\vspace{1mm}	% Some space before next section

% Publications

\section{Publications \hrulefill}
		
\vspace{2mm} %Some space

\runnormsubsection{Partial Multi-View Clustering Using Graph Regularized NMF}				\descript{Accepted}
\duration{Nishant Rai, Sumit Negi, Santanu Chaudhury, Om Deshmukh}{Jul '16}
\location{23rd International Conference on Pattern Recognition}

\vspace{2mm} %Some space

\runnormsubsection{Evolving structure of the Maritime Trade Network}				\descript{Accepted}
\duration{Zuzanna Kosowska-Stamirowska, César Ducruet, Nishant Rai}{May '16}
\location{Journal of Shipping and Trade}

\vspace{1mm}	% Some space before next section

% Internships

\section{Internships \hrulefill}
		
\vspace{2mm} %Some space

\runsubsection{C.A.R.I.S. Lab} 	\descript{Research Internship}
\duration{University of British Columbia}{May '16 - Jul '16}
\location{Vancouver, Canada}

\vspace{-2mm} %Reduce some space

	\begin{changemargin}{1.0cm}{0.0cm} 
	{
	\large{\bgemph{Single-Arm Reach Prediction:}}  \\
	\small{Mentored by \lgemph{Justin Hart, Post Doc, CARIS Lab} and \lgemph{Elizabeth Croft, Head, CARIS Lab}, for prediction of single-arm reaching motion by humans in order to create smooth and safe Human-Robot interactions.}
	\begin{tightitemize}
	\small
	{
	\item Worked on several robot platforms, including a \textbf{Barrett WAM} 7-DOF Robot and the \textbf{Willow Garage PR2} Robot.
	\item Trained in programming in the \textbf{ROS} (Robot Operating System) environment created by \textbf{Willow Garage}.
	\item Studied and analyzed the performance of multiple \textbf{Hand} and \textbf{Model} trackers and the possibility of inclusion in our pipeline.
	\item Developed and debugged interfaces in the experimental setup to be used in the \textbf{Human subject} Experiments.
	}
	\end{tightitemize}
	
\vspace{1mm} %Some space	

	\large{\bgemph{Merging Point Clouds from Multiple Kinects:}} \\
	\small{Mentored by \lgemph{Justin Hart, Post Doc, CARIS Lab} for aligning points clouds being received from multiple Kinects. Supporting project for improving the performance of other setups present in the lab.}
	\begin{tightitemize}
	\small
	{
	\item Literature survey on existing work for \textbf{Camera Calibration} and \textbf{Distortion reduction} in cameras.
	\item Performed \textbf{Chessboard Corner} detection to get target points to perform calibration. Computed the \textbf{Homography} from the detected points. Used parts of \textbf{Zhang's Camera Calibration} in order to extract the final camera parameters.
	\item Averaging using \textbf{Rodriguez} representation along with \textbf{Bundle Adjustment} used to improve results.
	\item \textbf{Transformation} between Kinects computed using the extracted camera parameters. Used to align the Point Clouds.
	}
	\end{tightitemize}
 	
	}
	\end{changemargin} 
	
	\vspace{1mm}
		
\runsubsection{X.R.C.I}				\descript{Research Internship}
\duration{Xerox Research Center India}{Dec '15}
\location{Bangalore, India}

\vspace{-2mm} %Reduce some space

	\begin{changemargin}{1.0cm}{0.0cm} 
	{
	\large{\bgemph{Multi View Clustering via Non Negative Matrix Factorization:}}  \\
	\small{Mentored by \lgemph{Om Deshmukh, Senior Researcher} (Area Manager, Multimedia Analytics), \lgemph{XRCI} and \lgemph{Sumit Negi, Principal Researcher, XRCI}, for developing and evaluating algorithms for Multi View Clustering using Non Negative Matrix factorization.}
	\begin{tightitemize}
	\small
	{ 	
	\item Literature Survey on \textbf{existing work} and \textbf{Variants} of Multi View Clustering; \textbf{Partial/Constrained Multi View Clustering}.
	\item Studied various algorithms for optimization including \textbf{Greedy Coordinate Descent}, \textbf{Alternating Least Squares}, Method of \textbf{alternate Optimizations}, \textbf{Augmented Lagrangian} methods. \textbf{Formulated update rules} for our methods based on them.
	\item Proposed, implemented and evaluated several models to tackle the Partial Multi View problem. \textbf{Outperform} existing models.
	\item Studied the effect of \textbf{Graph Regularization} on the results and the effect of \textbf{varying Kernels} on it, on multiple \textbf{Image} and \textbf{Textual} datasets. 	
	}
	\end{tightitemize}	
	}
	\end{changemargin} 	
	
	\vspace{1mm}
	
\runsubsection{I.N.R.I.A.}				\descript{Research Internship}
\duration{The French Institute for Research in Computer Science and Automation}{May '15 - Jul '15}
\location{Rocquencourt, France}
	
\vspace{-2mm} %Reduce some space

	\begin{changemargin}{1.0cm}{0.0cm} 
	{
	\large{\bgemph{Alternate Paths in Road networks:}}  \\
	\small{Mentored by \lgemph{Laurent Viennot, Senior Researcher, INRIA} and \lgemph{Adrian Kosowski, Researcher, INRIA}, for finding routes substantially different from the shortest path based on different criteria.} 
	\begin{tightitemize}
	\small
	{
	\item Implemented various \textbf{shortest path} algorithms and compared their efficiency on \textbf{real world road networks}.
	\item Proposed algorithms to compute paths according to another feasible definition.
	\item Created \textbf{measures} to \textbf{compare} different algorithms developed efficient algorithms for the involved computations.
	}
	\end{tightitemize}
	
\vspace{1mm} %Some space	
			
	\large{\bgemph{Feature Based Representation of Social Networks:}} \\
	\small{Mentored by \lgemph{Adrian Kosowski, Researcher, INRIA}, finding good local features which are suitable predictors for global features}
	\begin{tightitemize}
	\small
	{
	\item Studied \textbf{information spread models} and about maximizing spread, \textbf{Local Ranking} problem, \textbf{Pagerank} algorithm. 
	\item Implemented and studied randomized \textbf{rumor spreading}, the relation between size and steps for spread of the rumor
	\item Studied and explored different \textbf{local features} in graphs based on \textbf{walks, subgraph densities, centrality measures} and their relation with other \textbf{global properties} along with arguments to explain the obtained results.
	}
	\end{tightitemize}
 	
	}
	\end{changemargin} 	

\vspace{0mm}	% Some space before next section

% Details about the projects

\section{Projects \hrulefill}

\vspace{2mm} %Some space

	\runsubsection{ChaLearn: Apparent Personality Analysis through Videos:}
	\descript{Jun '16 - Jul '16}
	\begin{tightitemize}
	\small
	{
	\item Project for challenge hosted for \textbf{ChaLearn Looking at People, ECCV '16 workshop}. Secured \textbf{6\textsuperscript{th}} place amongst \textbf{85} teams.
	\item Challenge aimed at recognizing \textbf{personality traits} of users in \textbf{short video} sequences (15 secs).
	\item Our method titled \textbf{"Multi-modal Approaches for Personality Analysis through Videos"} constructs multiple models using different modalities and performs \textbf{late feature fusion} in order to predict the traits.
	\item Preprocessing of videos involves \textbf{Face detection} followed by \textbf{Facial Landmark} alignment.
	\item Experimented with \textbf{multiple} visual feature based models including \textbf{Background-Context} and \textbf{Facial feature} models.
	\item \textbf{Visual features} computed using \textbf{CNNs} trained on video frames. Also experimented with activations of the last layer of \textbf{VGG-Net}.
	\item \textbf{Audio features} used same as the one used in \textbf{INTERSPEECH 2010} Paralinguistic Challenge. Audio clips broken into \textbf{multiple segments}. Experimented with various \textbf{pooling} methods and \textbf{regressors} for the final prediction.
	\item Multiple methods used for \textbf{combining} the \textbf{frame-wise predictions} to extract the final results. More details available in \textbf{report} online.
	\item \textbf{Stacking} and \textbf{Blending} performed to fuse the Multi-Modal predictions. Results \textbf{much better} than models using a single modality.
	}
	\end{tightitemize}

	\vspace{2mm}

	\runsubsection{Deep Learning for Visual Question Answering:}				
	\descript{Mar '16 - Apr '16}	
	\begin{tightitemize}
	\small
	{
	\item \textbf{Course Project} for course \textbf{CS676A}: Computer Vision and Image Processing, under \textbf{Prof. Vinay Namboodiri}.
	\item Project aimed at constructing Neural Network-based models for \textbf{Answering Open-ended questions} about images.
	\item Task reduced to \textbf{multi-class classification} after selecting the top K answers from the training set, and restricting the final output.
	\item Image embeddings are activations of the last layer of a \textbf{Convolutional Neural Network} (pre-trained on the \textbf{ImageNet} dataset).
	\item Question embeddings are computed using the \textbf{Glove Word Vectors} of the constituent words and \textbf{LSTM} Networks.
	\item \textbf{Multiple methods} used for combining the embeddings and their effect on the performance was studied. The entire network (except the CNN for image features) is trained \textbf{End-to-End}.
	\item Studied about \textbf{Spatial-Attention} and \textbf{External Knowledge} Based Models for Visual Question Answering. Implemented a \textbf{Semantic-Attention} model, where the attention cues are extracted using the question.
	}
	\end{tightitemize}

	\vspace{2mm}

	\runsubsection{Real-Time Vehicle and License Plate Recognition:}				
	\descript{Feb '16 - Apr '16}	
	\begin{tightitemize}
	\small
	{
	\item \textbf{Course Project} for course \textbf{CS771A}: Machine Learning: Tools and Techniques, under \textbf{Prof. Harish Karnick}.
	\item Project aimed at \textbf{Real-time vehicle Recognition} along with \textbf{Extracting Registration Numbers} from the \textbf{License Plates} of four-wheelers in real world \textbf{surveillance} videos.
	\item Proposed method involves processing the video frames to extract \textbf{candidate regions} containing vehicles. \textbf{Tracking} performed using \textbf{inter-frame information} and \textbf{SIFT-based interest point matching}.
	\item Performed classification on the proposed regions using \textbf{SVMs, Random Forests, CNNs} and \textbf{Ensembles} (Boosted through \textbf{AdaBoost}).
	\item Studied the effect of the different features (Such as \textbf{HOG, SIFT} and \textbf{CNN} based features) on the performance.
	\item License plate \textbf{localized} after processing the candidate region. \textbf{OpenALPR} used to further narrow and extract the Vehicle number.
	}
	\end{tightitemize}

	\vspace{2mm}

	\runsubsection{ADA Compiler:}
	\descript{Jan '16 - Apr '16}	
	\begin{tightitemize}
	\small
	{
	\item \textbf{Course Project} for the completion of the course \textbf{CS335A}: Compiler Design, under \textbf{Prof. Subhajit Roy}.
	\item Project involved creating an \textbf{End-to-End Compiler} for a subset of the programming language \textbf{ADA} in the \textbf{x86} architecture.
	\item Implemented a \textbf{Lexical Analyzer} and \textbf{Assembly-Code Generator} in python, constructed grammar rules for parsing our identified language and created the \textbf{TAC} (Three Address Code) for intermediate code. Used \textbf{Yacc} and \textbf{Lex} for the same.
	\item Implemented basic types, operations for \textbf{Strings}, \textbf{Library support}, \textbf{Short circuiting}, conditionals, \textbf{Loops} with strict \textbf{type-checking} and error handling. Implemented \textbf{functions} (Allowed \textbf{overloading}) with multiple return values and scopes.	
	}
	\end{tightitemize}

	\vspace{2mm}

	\runsubsection{Adaptive Strategies for Infinite Prisoner's Dilemma:}
	\descript{Jan '16 - Apr '16}	
	\begin{tightitemize}
	\small
	{
	\item \textbf{Case Study} for the completion of the course \textbf{ECO502A}: Applied Game Theory, under \textbf{Prof. Vimal Kumar}.
	\item Studied existing literature on the work related to \textbf{Prisoner's Dilemma} and the analysis of the Infinite Case.
	\item Implemented and studied the performance of both Single-Objective and Multi-Objective \textbf{Evolutionary Algorithms} to compute good strategies for the same. Confirm results proposed by \textbf{Axelrod} by conducting tournaments amongst multiple strategies.
	\item Proposed new \textbf{adaptive} strategies based on \textbf{Reinforcement Learning} and study its performance by conducting various experiments.
	\item Studied about methods such as \textbf{Q-Learning, Deep Reinforcement Learning} and its possible applications in learning new strategies.
	}
	\end{tightitemize}

	\vspace{2mm}


	\runsubsection{Word Embeddings with Multiple Word prototypes:}				\descript{Aug '15 - Nov '15}	
	\begin{tightitemize}
	\small
	{
	\item \textbf{Course Project} for course \textbf{CS671A}: Introduction to Natural Language Processing, under \textbf{Prof. Amitabha Mukherjee}.
	\item Project aimed at constructing of \textbf{Multiple Sense Embeddings} for different words using purely \textbf{unsupervised approaches}.
	\item Proposed algorithms involved \textbf{Online clustering}, analysis of Word-Word \textbf{co-occurrence matrix} and \textbf{Non-parametric clustering} using penalties based on \textbf{Negative Sampling}.
	\item \textbf{Outperform existing methods} in \textbf{Local Similarity} Metric and \textbf{comparable} in terms of other metrics, result in more semantically coherent senses than the state of the art methods.
	}
	\end{tightitemize}
		
	\vspace{2mm}

	\runsubsection{NachOS Operating System:}   \descript{July '15 - Nov '15}
	\begin{tightitemize}
	\small
	{
	\item \textbf{Course Project} for course \textbf{CS330A}: Operating Systems, under \textbf{Prof. Mainak Chaudhuri}.
	\item \textbf{Extended} the \textbf{NachOS} operating system to perform basic operating system functions including \textbf{Fork, Join, Sleep} and \textbf{Exec}. 
	\item Implemented and evaluated performance of various algorithms for scheduling processes.
	\item Developed and added support for \textbf{Demand Paging, Shared Memory, Condition Variables} and \textbf{Semaphores}.	
	}
	\end{tightitemize}
		
	\vspace{2mm}
	
	\runsubsection{Multi Modal Emotion Recognition:}   \descript{May '14 - Jun '14}
	\begin{tightitemize}
	\small
	{
	\item Project aimed at performing \textbf{Emotion Detection} using three features i.e. \textbf{textual, speech} and \textbf{visual}.
	\item Recognition of facial expressions using \textbf{Eigenfaces}. Further narrowed down the features by detecting important parts such as Eyes, Nose, etc using \textbf{Haar Cascade} Classifiers.
	\item Used \textbf{acoustic features} of audio such as \textbf{Mel Frequency Cepstral Coefficients (MFCC)} to extract sentiment out of speech.
	\item Merged the results of the three classifiers to identify emotions more accurately.
	\item Learnt about \textbf{Facial Action Codings, Active Shape/Appearance Models} and other prevalent methods for emotion classification.
	}
	\end{tightitemize}
		
	\vspace{2mm}
	
	\runsubsection{Geometric Data Structures:}   \descript{Sep '14 - Nov '14}
	\begin{tightitemize}
	\small
	{
	\item Project for \textbf{Advanced Track} in course \textbf{CS210}: Data Structures and Algorithms, under \textbf{Prof. Surendar Baswana}.
	\item Project involved \textbf{re-invention, implementation} and \textbf{analysis} of \textbf{geometric} data structures to efficiently answer given queries.
	\item Developed efficient algorithms for maintaining \textbf{Dynamic Convex Hulls} and \textbf{Range Search Queries}.
	\item \textbf{Queries handled:} \textbf{Point in Polygon, Polygon-Line intersection, Simplex problem, Orthogonal Range Search, Half Plane problem}.
	}
	\end{tightitemize}
 		
	\vspace{2mm}
			
	\runsubsection{Sentiment Analysis of Social Media:}   \descript{Aug' 14}
	\begin{tightitemize}
	\small
	{
	\item Application developed during \textbf{Web-Dev, Takneek '14} and secured \textbf{First} position.
	\item Interface to analyze the past and present \textbf{social sentiment of brands} and \textbf{their products}.
	\item Identifies the \textbf{"good" and "bad" features} of the product to act upon them.
	}
	\end{tightitemize}
 		
	\vspace{2mm}
	
	\runsubsection{Other Minor Projects:}		\descript{\null}
	\begin{tightitemize}
	\small
	{
	\item Developed a bot to play \textbf{Othello}, based on \textbf{Minimax} algorithm. \textbf{Alpha-beta Pruning} performed to speed up computations. Secured \textbf{19\textsuperscript{th}} place amongst \textbf{2000+} participants from over the world.
	\item Designed a bot to play \textbf{Battleship}. A \textbf{probabilistic model} of the ships and board used to decide the next move.
	\item Created models for \textbf{Predicting Search trends}, \textbf{Topic Assignment} based on keywords, \textbf{Spam Detection} and \textbf{Multi-Label Question} classification (Tested on questions taken from \textbf{Quora}).
	\item \textbf{News Report Classification} completed during {\textbf{Jan '14 - Apr '14}} under \textbf{Association of Computing Activities}.
	\item Designed a \textbf{Captcha Decoder}, able to work with occlusions. \textbf{Clustering} and \textbf{Segmentation} based methods used for extracting \textbf{candidate regions} containing characters. Further passed through a classifier for confirmation.
	\item Completed project to discover patterns and trends about the \textbf{New York Subway}, under the \textbf{Udacity} course: \textbf{Intro to Data Science}.
	}
	\end{tightitemize}		

\vspace{1mm}	% Some space before next section

% Skills

\section{Technical Skills \hrulefill}

\vspace{1mm}	% Some space

\category{\normalsize{Programming Languages} \small\scshape(Proficient):} \hfill \emphasize{C, C++, Python, MatLab, GNU Octave, Assembly (Verilog)} \\
\vspace{1mm}	%Some Space
\category{\normalsize{Programming Languages} \small\scshape(Familiar):} \hfill \emphasize{Java, CSS, JavaScript, PHP, MySQL} \\
\vspace{1mm}	%Some Space
\category{\normalsize{Software and Utilities:}} \hfill \emphasize{Git, GNUPlot, \LaTeX\, AUTOCAD Inventor}

\vspace{1mm}	% Some space before next section

% Interests

\section{Interests \hrulefill}

\vspace{1mm} %Some space

	{\large{
	{Algorithms and Data Structures \hfill Artificial Intelligence \\
	Competitive Programming\hfill Computer Vision\\
	Machine Learning \hfill Natural Language Processing\\}		
	}}

\vspace{1mm}	% Some space before next section

% Positions of Responsibility

\section{Positions of Responsibility \hrulefill}

\vspace{2mm} %Some space
		
	\begin{tabular}{r|p{16cm}}	
	
	\null \hspace{10mm} \normalsize\emph{Jan '15 - Apr '16} & \textbf{Member, Core Team Academics, Counseling Service}\\
	& \small{Responsible for managing remedial lectures, mentor allotment and other academics related issues.}\\
	& \small{Managing a team of 100+ Academic Mentors to help and guide academically troubled students.}\\
	& \small{Assisting peer students in departmental courses by conducting classes as well as personal tutoring.}\\

	\multicolumn{2}{c}{} \\
	\normalsize\emph{Aug '14 - Mar '15} & \textbf{Senior Executive, Public Relations, Techkriti '15, IIT Kanpur}\\
	& \small{Responsible for inviting eminent personalities for talks, shows and looking after their publicity and hospitality}\\
	& \small{Responsible for smooth conduction of 5 talks and 2 shows in the festival along with other teammates.}\\
	& \small{Managed a team of 15 members to organise TechPlanet which witnessed footfall of over 3000 people}	\\

	\multicolumn{2}{c}{} \\
	\normalsize\emph{Apr '14 - Apr '15} & \normalsize\textbf{Secretary, Programming Club}\\

	\normalsize\emph{Jun '14 - Apr '15} & \normalsize\textbf{Academic Mentor (MTH101/102), Counselling Service}\\ 

	\normalsize\emph{Jun '14 - Apr '15} & \normalsize\textbf{Student Guide, Counselling Service}\\

	\normalsize\emph{Previous} & \normalsize\textbf{Secretary, Hospitality Cell, Udghosh '14}\\
	
	& \normalsize\textbf{Volunteer, Hospitality Cell, Udghosh '13}\\

	\end{tabular}

\vspace{2mm}	% Some space before next section

% Relevant Courses

\section{Relevant Courses \hrulefill}

\vspace{2mm} %Some space
		
\begin{small}
	\begin{tabularx}{\textwidth}{X X l}
	{\textbf{ESC101} : Fundamentals of Computing} & {\textbf{MTH101} : Analytical Calculus} &{\textbf{MTH102} : Linear Algebra and DE}\\
	{\textbf{CS201} : Discrete Mathematics}&{\textbf{CS210 }: Data Structures and Algorithms} & {\textbf{Coursera}: Algorithms }\\
	{\textbf{Udacity}: Intro to Data Science } & {\textbf{CS251} : Computing Laboratory} &{\textbf{CS202} : Mathematical Logic} \\ 
	{\textbf{CS203} : Abstract Algebra} & {\textbf{CS220} : Computer Organization} & {\textbf{MSO201} : Probability and Statistics}\\
	{\textbf{CS345} : Algorithms - II}&{\textbf{CS330} : Operating Systems}&{\textbf{CS340} : Theory of Computation}\\
	{\textbf{CS252} : Computing Laboratory - II}&{\textbf{CS671} : Natural Language Processing} &{\textbf{CS771} : \scriptsize{Machine Learning: Tools and Techniques}}\\
	{\textbf{CS676} : \scriptsize{Computer Vision and Image Processing}}&{\textbf{CS335} : Compiler Design}
	& {\hfill \footnotesize\emphasize{* - Ongoing}}\\
	\end{tabularx}
\end{small}	

\vspace{2mm}	% Some space before next section

% Relevant Courses

\section{Extra-Curricular Activities \hrulefill}

\vspace{2mm} %Some space
	
	Secured \textbf{1\textsuperscript{st}} place in \textbf{Reviews} - Lifestyle event in \textbf{Spectrum '14} (Inter-Hostel Competition).\\
	Secured \textbf{1\textsuperscript{st}} place in \textbf{Tennis (Singles)} in \textbf{Freshers' Inferno '13} (Inter-Hostel Sports Competition).\\	
	Selected for \textbf{CBSE Tennis Regionals} (Guwahati Region), 2009.\\
	Selected for participation in \textbf{U-17 Inter School Table-Tennis Tournament}, Shillong in 2010.\\
	Secured \textbf{1\textsuperscript{st}} place in \textbf{U-17 Tennis Open} held at Basava International School, New Delhi in 2011.\\

\end{document}