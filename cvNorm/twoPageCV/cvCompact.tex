\documentclass[a4paper]{norm-resume}

\usepackage{array}
\usepackage{tabularx}
\usepackage{setspace}

\def\changemargin#1#2{\list{}{\rightmargin#2\leftmargin#1}\item[]}
\let\endchangemargin=\endlist 

\pagenumbering{gobble}

\begin{document}

%----------------------------------------------------------------------------------------
%	TITLE SECTION
%----------------------------------------------------------------------------------------

\lastupdated % Print the Last Updated text at the top right

\namesection{Nishant}{Rai}{}{Email: nishantr@iitk.ac.in}

\vspace{2mm}

% Education section : With current and previous schooling

\section{Education \hrulefill}
		
\begin{tabularx}{\textwidth}{c l l r}	
	{April 2017 \hspace{10mm}} & B.Tech \emphasize{(Computer Science And Engineering)} \hspace{22mm} & \emphasize{IIT Kanpur \hspace{10mm}} & {\textbf{{9.9/10.0}}}\\		
	{April 2013 \hspace{10mm}} & Class XII \emphasize{(Central Board for Senior Education)} & \emphasize{K.V. Delhi} & {\textbf{{96.20 \%}}}\\
	{April 2011 \hspace{10mm}} & Class X \emphasize{(Central Board for Senior Education)} & \emphasize{K.V. Shillong} & {\textbf{{10.0/10.0}}}\\	
\end{tabularx}

\vspace{1mm}	% Some space before next section

% Academic Acheivements

\section{Academic Achievements \hrulefill}

\vspace{3mm} %Some space
		
\begin{tightitemize}
	\item Received the \textbf{Academic Excellence Award} for exceptional academic performance in \textbf{13-14} and \textbf{14-15} academic session.
	\item Secured AIR - \textbf{257} in {\textbf{JEE (Advanced) 2013}} and AIR - {\textbf{79}} in \textbf{JEE (Mains) 2013}.
	\item One amongst the \textbf{6 INMO Awardees} selected for \textbf{IMO training camp}, after clearing \textbf{INMO '13}.
	\item Awarded Gold medal for being selected for \textbf{IPhO training camp}, after clearing \textbf{INPhO '13}.
	\item Selected for the prestigious \textbf{KVPY} scholarship, in stream SX.
\end{tightitemize}
	
\vspace{0mm}	% Some space before next section

% Programming Acheivements

\section{Achievements in Programming \hrulefill}

\vspace{3mm} %Some space

\begin{tightitemize}
	\item Secured \textbf{1\textsuperscript{st}} place in \textbf{Microsoft CodeHunt} amongst individuals from over the country. Selected for \textbf{finals} in \textbf{China}.
	\item Secured \textbf{13\textsuperscript{th}} place in \textbf{IOPC} amongst \textbf{900+} teams from over the world held during \textbf{Techkriti '15}.
	\item Secured \textbf{8\textsuperscript{th}} place in \textbf{ZS Associates Data Science Challenge} amongst \textbf{300+} students from over the country.
	\item Secured \textbf{26\textsuperscript{th}} place in \textbf{ACM ICPC Onsite Contest} 2014 amongst \textbf{250+} teams from over the country.
\end{tightitemize}

\vspace{1mm}	% Some space before next section

% Internships

\section{Internships \hrulefill}
		
\vspace{2mm} %Some space

\runsubsection{C.A.R.I.S. Lab} 	\descript{Research Internship}
\duration{University of British Columbia, Vancouver, Canada}{May '16 - Jul '16}
\location{}

\vspace{-3mm} %Reduce some space

	\begin{changemargin}{0.7cm}{0.0cm} 
	{
	\large{\bgemph{Single-Arm Reach Prediction:}}  \\
	\small{Mentored by \lgemph{Justin Hart, Post Doc, CARIS Lab} and \lgemph{Elizabeth Croft, Head, CARIS Lab}, for prediction of \textbf{single-arm reaching} motion by humans in order to create smooth and safe Human-Robot interactions.}

	\vspace{-1mm} %Reduce some space

	\large{\bgemph{Merging Point Clouds from Multiple Kinects:}} \\
	\small{Mentored by \lgemph{Justin Hart, Post Doc, CARIS Lab} for \textbf{aligning points clouds} being received from \textbf{multiple Kinects}. Supporting project for improving the performance of other setups present in the lab.}

	}
	\end{changemargin} 
			
\runsubsection{X.R.C.I}				\descript{Research Internship}
\duration{Xerox Research Center India, Bangalore, India}{Dec '15}
\location{}

\vspace{-3mm} %Reduce some space

	\begin{changemargin}{0.7cm}{0.0cm} 
	{
	\large{\bgemph{Multi View Clustering via Non Negative Matrix Factorization:}}  \\
	\small{Mentored by \lgemph{Om Deshmukh, Senior Researcher} (Area Manager, Multimedia Analytics), \lgemph{XRCI} and \lgemph{Sumit Negi, Principal Researcher, XRCI}, for developing and evaluating algorithms for Multi View Clustering using Non Negative Matrix factorization.}
	}
	\end{changemargin} 	
		
\runsubsection{I.N.R.I.A.}				\descript{Research Internship}
\duration{French Institute for Research in Computer Science, Rocquencourt, France}{May '15 - Jul '15}
\location{}
	
\vspace{-3mm} %Reduce some space

	\begin{changemargin}{0.7cm}{0.0cm} 
	{
	\large{\bgemph{Alternate Paths in Road networks:}}  \\
	\small{Mentored by \lgemph{Laurent Viennot, Senior Researcher, INRIA} and \lgemph{Adrian Kosowski, Researcher, INRIA}, for finding routes substantially different from the shortest path based on different criteria.} 
	
\vspace{-1mm} %Some space	
			
	\large{\bgemph{Feature Based Representation of Social Networks:}} \\
	\small{Mentored by \lgemph{Adrian Kosowski, Researcher, INRIA}, finding good local features which are suitable predictors for global features}

	}
	\end{changemargin} 	

\vspace{-2mm}	% Some space before next section

% Details about the projects

\section{Projects \hrulefill}

\vspace{2mm} %Some space

	\runsubsection{ChaLearn: Apparent Personality Analysis through Videos:}
	\descript{Jun '16 - Jul '16}
	\begin{tightitemize}
	\small
	{
	\item Project for challenge hosted for \textbf{ChaLearn Looking at People, ECCV '16 workshop}. Secured \textbf{6\textsuperscript{th}} place amongst \textbf{85} teams.
	\item Preprocessing of videos involves \textbf{Face detection} followed by \textbf{Facial Landmark} alignment.
	\item Experimented with multiple \textbf{audio} and \textbf{visual} feature based models including \textbf{Background-Context} and \textbf{Facial feature} models.
	\item Multiple methods used for \textbf{combining} the \textbf{frame-wise predictions} to extract the final results. More details available in \textbf{report} online.
	\item \textbf{Stacking} and \textbf{Blending} performed to fuse the Multi-Modal predictions. Results \textbf{much better} than models using a single modality.
	}
	\end{tightitemize}

	\vspace{2mm}

	\runsubsection{Deep Learning for Visual Question Answering:}				
	\descript{Mar '16 - Apr '16}	
	\begin{tightitemize}
	\small
	{
	\item \textbf{Course Project} for course \textbf{CS676A}: Computer Vision and Image Processing, under \textbf{Prof. Vinay Namboodiri}.
	\item Project aimed at constructing Neural Network-based models for \textbf{Answering Open-ended questions} about images.
	\item Task reduced to \textbf{multi-class classification} after selecting the top K answers from the training set, and restricting the final output.
	\item \textbf{Multiple methods} used for combining the visual and audio features, and their effect on the performance was studied. The entire network (except the CNN for image features) is trained \textbf{End-to-End}.
	}
	\end{tightitemize}

	\vspace{2mm}

	\runsubsection{Real-Time Vehicle and License Plate Recognition:}				
	\descript{Feb '16 - Apr '16}	
	\begin{tightitemize}
	\small
	{
	\item \textbf{Course Project} for course \textbf{CS771A}: Machine Learning: Tools and Techniques, under \textbf{Prof. Harish Karnick}.
	\item Project aimed at \textbf{Real-time vehicle Recognition} along with \textbf{Extracting Registration Numbers} from \textbf{License Plates}.
	\item Proposed method involves processing the video frames to extract \textbf{candidate regions} containing vehicles.
	\item License plate \textbf{localized} after processing the candidate region. \textbf{OpenALPR} used to further narrow and extract the Vehicle number.
	}
	\end{tightitemize}

	\vspace{2mm}

	\runsubsection{ADA Compiler:}
	\descript{Jan '16 - Apr '16}	
	\begin{tightitemize}
	\small
	{
	\item \textbf{Course Project} for the completion of the course \textbf{CS335A}: Compiler Design, under \textbf{Prof. Subhajit Roy}.
	\item Project involved creating an \textbf{End-to-End Compiler} for a subset of the programming language \textbf{ADA} in the \textbf{x86} architecture.
	}
	\end{tightitemize}

	\vspace{2mm}

	\runsubsection{Adaptive Strategies for Infinite Prisoner's Dilemma:}
	\descript{Jan '16 - Apr '16}	
	\begin{tightitemize}
	\small
	{
	\item \textbf{Case Study} for the completion of the course \textbf{ECO502A}: Applied Game Theory, under \textbf{Prof. Vimal Kumar}.
	\item Studied existing literature on the work related to \textbf{Prisoner's Dilemma} and the analysis of the Infinite Case.
	\item Implemented and studied the performance of \textbf{Evolutionary Algorithms} to compute good strategies for the same.
	\item Proposed new \textbf{adaptive} strategies based on \textbf{Reinforcement Learning} and study its performance by conducting various experiments.
	}
	\end{tightitemize}

	\vspace{2mm}

	\runsubsection{Word Embeddings with Multiple Word prototypes:}				\descript{Aug '15 - Nov '15}	
	\begin{tightitemize}
	\small
	{
	\item \textbf{Course Project} for course \textbf{CS671A}: Introduction to Natural Language Processing, under \textbf{Prof. Amitabha Mukherjee}.
	\item Project aimed at constructing of \textbf{Multiple Sense Embeddings} for different words using purely \textbf{unsupervised approaches}.
	\item \textbf{Outperform existing methods} in \textbf{Local Similarity} Metric and \textbf{comparable} in terms of other metrics, result in more semantically coherent senses than the state of the art methods.
	}
	\end{tightitemize}
		
	\vspace{2mm}

	\runsubsection{NachOS Operating System:}   \descript{July '15 - Nov '15}
	\begin{tightitemize}
	\small
	{
	\item \textbf{Course Project} for course \textbf{CS330A}: Operating Systems, under \textbf{Prof. Mainak Chaudhuri}.
	\item \textbf{Extended} the \textbf{NachOS} operating system to perform basic operating system functions including \textbf{Fork, Join, Sleep} and \textbf{Exec}. 
	\item Implemented and evaluated performance of various algorithms for scheduling processes.
	}
	\end{tightitemize}
		
	\vspace{2mm}
	
	\runsubsection{Multi Modal Emotion Recognition:}   \descript{May '14 - Jun '14}
	\begin{tightitemize}
	\small
	{
	\item Project aimed at performing \textbf{Emotion Detection} using multiple features i.e. \textbf{textual, speech} and \textbf{visual}.
	\item Experimented with multiple types of \textbf{facial features} and their effect on performance.
	\item Used \textbf{acoustic features} of audio such as \textbf{Mel Frequency Cepstral Coefficients (MFCC)} to extract sentiment out of speech.
	\item \textbf{Merged} the results of the three classifiers to identify emotions more accurately.
	}
	\end{tightitemize}
		
	\vspace{2mm}
	
	\runsubsection{Geometric Data Structures:}   \descript{Sep '14 - Nov '14}
	\begin{tightitemize}
	\small
	{
	\item Project for \textbf{Advanced Track} in course \textbf{CS210}: Data Structures and Algorithms, under \textbf{Prof. Surendar Baswana}.
	\item Project involved \textbf{re-invention, implementation} and \textbf{analysis} of \textbf{geometric} data structures to efficiently answer given queries.
	\item Queries handled: \textbf{Point in Polygon, Polygon-Line intersection, Simplex problem, Orthogonal Range Search, Half Plane problem}.
	}
	\end{tightitemize}
 		
	\vspace{2mm}
				
	\runsubsection{Other Minor Projects:}		\descript{\null}
	\begin{tightitemize}
	\small
	{
	\item Developed an application for \textbf{Sentiment Analysis} of feed from \textbf{social media} during \textbf{Web-Dev, Takneek '14}  and secured \textbf{1\textsuperscript{st}} place.
	\item Developed a bot to play \textbf{Othello}, based on \textbf{Minimax} algorithm. \textbf{Alpha-beta Pruning} performed to speed up computations. Secured \textbf{19\textsuperscript{th}} place amongst \textbf{2000+} participants from over the world.
	\item Designed a bot to play \textbf{Battleship}. A \textbf{probabilistic model} of the ships and board used to decide the next move.
	\item Created models for \textbf{Predicting Search trends}, \textbf{Topic Assignment} based on keywords, \textbf{Spam Detection} and \textbf{Multi-Label Question} classification (Tested on questions taken from \textbf{Quora}).
	\item Designed a \textbf{Captcha Decoder}, able to work with occlusions. \textbf{Clustering} and \textbf{Segmentation} based methods used for extracting \textbf{candidate regions} containing characters. Further passed through a classifier for confirmation.
	}
	\end{tightitemize}

\vspace{1mm}	% Some space before next section

% Skills

\section{Technical Skills \hrulefill}

\vspace{1mm}	% Some space

\category{\normalsize{Programming Languages} \small\scshape(Proficient):} \hfill \emphasize{C, C++, Python, MatLab, GNU Octave, Assembly (Verilog)} \\
\vspace{1mm}	%Some Space
\category{\normalsize{Programming Languages} \small\scshape(Familiar):} \hfill \emphasize{Java, CSS, JavaScript, PHP, MySQL} \\
\vspace{1mm}	%Some Space
\category{\normalsize{Software and Utilities:}} \hfill \emphasize{Git, GNUPlot, \LaTeX\, AUTOCAD Inventor}

\vspace{1mm}	% Some space before next section

% Positions of Responsibility

\section{Positions of Responsibility \hrulefill}

\vspace{2mm} %Some space
		
	\begin{tabular}{r|p{16cm}}	

	\null \hspace{10mm} \normalsize\emph{Jan '15 - Apr '16} & \textbf{Member, Core Team Academics, Counseling Service}\\

	\normalsize\emph{Aug '14 - Mar '15} & \textbf{Senior Executive, Public Relations, Techkriti '15, IIT Kanpur}\\

	\normalsize\emph{Apr '14 - Apr '15} & \normalsize\textbf{Secretary, Programming Club}\\

	\normalsize\emph{Jun '14 - Apr '15} & \normalsize\textbf{Academic Mentor (MTH101/102), Counselling Service}\\ 

	\normalsize\emph{Jun '14 - Apr '15} & \normalsize\textbf{Student Guide, Counselling Service}\\

	\end{tabular}

\vspace{2mm}	% Some space before next section

\section{Extra-Curricular Activities \hrulefill}

\vspace{2mm} %Some space
	
	Secured \textbf{1\textsuperscript{st}} place in \textbf{Reviews} - Lifestyle event in \textbf{Spectrum '14} (Inter-Hostel Competition).\\
	Secured \textbf{1\textsuperscript{st}} place in \textbf{Tennis (Singles)} in \textbf{Freshers' Inferno '13} (Inter-Hostel Sports Competition).\\	

\end{document}